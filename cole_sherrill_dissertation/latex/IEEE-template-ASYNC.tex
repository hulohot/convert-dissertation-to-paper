\documentclass[conference]{IEEEtran}
\IEEEoverridecommandlockouts
% The preceding line is only needed to identify funding in the first footnote. If that is unneeded, please comment it out.
%Template version as of 6/27/2024

\usepackage{cite}
\usepackage{amsmath,amssymb,amsfonts}
\usepackage{algorithmic}
\usepackage{graphicx}
\usepackage{textcomp}
\usepackage{xcolor}

\begin{document}

\title{Synthesis and Timing Constraint Generation for Reliable MTNCL Asynchronous Circuits}

\author{\IEEEauthorblockN{Cole Sherrill}
\IEEEauthorblockA{\textit{Department of Computer Science and Computer Engineering} \\
\textit{University of Arkansas}\\
Fayetteville, AR, USA \\
csherril@uark.edu}
}

\maketitle

\begin{abstract}
This paper presents an automated synthesis flow for Multi-Threshold NULL Convention Logic (MTNCL) asynchronous circuits that enables reliable implementation while maximizing performance. The flow transforms conventional RTL descriptions into optimized MTNCL implementations through a series of automated stages including library preparation, single-rail synthesis, dual-rail expansion, and registration/handshaking insertion. Novel timing constraints are integrated into the flow to prevent race conditions while enabling performance optimization. The methodology is validated through implementation of a Montgomery Multiplier, demonstrating significant improvements in performance and energy efficiency compared to structural designs while maintaining reliability. Results show up to X\% improvement in cycle time and Y\% reduction in energy consumption when applying the proposed flow.
\end{abstract}

\begin{IEEEkeywords}
Asynchronous circuits, MTNCL, synthesis methodology, timing constraints, automation
\end{IEEEkeywords}

\section{Introduction}
Multi-Threshold NULL Convention Logic (MTNCL) offers compelling advantages for asynchronous circuit design, including reduced power consumption through fine-grained sleep mode and robust operation across process variations. However, widespread adoption has been limited by the lack of automated synthesis methodologies that can reliably transform conventional RTL designs into optimized MTNCL implementations.

This paper presents a comprehensive synthesis flow that addresses this challenge through:
\begin{itemize}
\item Automated transformation from single-rail RTL to dual-rail MTNCL
\item Systematic optimization of logic, completion detection, and handshaking
\item Integration with commercial EDA tools and standard cell methodologies
\item Essential timing constraints to ensure reliable operation
\end{itemize}

The key contributions of this work include:
\begin{itemize}
\item Development of a complete RTL-to-layout synthesis flow for MTNCL circuits
\item Novel optimization techniques for dual-rail expansion and completion detection
\item Integration of timing constraints for race prevention and performance optimization
\item Validation through implementation of a Montgomery Multiplier case study
\end{itemize}

The remainder of this paper is organized as follows: Section II provides background on MTNCL timing sensitivities. Section III presents the detailed synthesis flow methodology. Section IV describes essential timing constraints. Section V demonstrates the effectiveness of the flow through experimental results, and Section VI concludes the paper.

\section{Background and Motivation}
Multi-Threshold NULL Convention Logic (MTNCL) circuits require careful consideration of timing dependencies to ensure reliable operation. While MTNCL is fundamentally a quasi-delay-insensitive (QDI) methodology, the incorporation of sleep mode functionality introduces potential race conditions that must be properly constrained. This section analyzes the key timing sensitivities and race conditions that can impact circuit reliability.

\subsection{DATA Completion Race}
The DATA completion race occurs during the transition from NULL to DATA. When sleep is de-asserted, the circuit begins evaluating DATA inputs. However, if the completion detection logic evaluates too quickly relative to the combinational logic computation, false completion may be signaled before valid DATA propagates through the entire circuit.

\subsection{DATA Handshaking Race} 
During DATA evaluation, proper handshaking between pipeline stages is critical. The DATA handshaking race manifests when a receiving stage acknowledges DATA before it has been properly captured by all registration elements. This can lead to premature NULL insertion and data corruption.

\subsection{NULL Completion Race}
Similar to DATA completion, the NULL completion race involves the timing relationship between combinational logic reset and completion detection during the transition from DATA to NULL. Early completion detection during this phase can cause the next DATA wave to begin evaluation before the circuit has fully reset.

\subsection{NULL Handshaking Race}
The NULL handshaking race occurs during the reset phase when a receiving stage acknowledges NULL before complete propagation through its registration elements. This can result in premature DATA evaluation in the previous stage, potentially creating metastability or data hazards.

\subsection{Limitations of Existing Flows}
Traditional synthesis flows lack the capability to properly analyze and constrain these race conditions. Most commercial tools are optimized for synchronous designs and do not provide native support for modeling MTNCL's dual-rail encoding, completion detection, and handshaking protocols. Manual implementation of these constraints is error-prone and does not scale well with circuit complexity.

\section{Synthesis Flow}
This section presents a comprehensive RTL-to-layout synthesis flow for MTNCL circuits that addresses the reliability challenges while enabling automated implementation. The flow transforms a conventional single-rail RTL description into a fully-constrained MTNCL implementation through a series of automated stages, each building upon established commercial EDA tool capabilities.

\subsection{Library Preparation}
The foundation of the synthesis flow is a characterized MTNCL standard cell library. Key components include:

\begin{itemize}
\item Threshold gates with sleep capability
  \begin{itemize}
  \item TH22, TH33, TH44 gates with varying thresholds
  \item Sleep transistor sizing for optimal power gating
  \item Characterized delays for both DATA and NULL propagation
  \end{itemize}
\item Completion detection components
  \begin{itemize}
  \item Multi-input completion trees
  \item Local and global completion networks
  \item Optimized for minimum detection latency
  \end{itemize}
\item Registration elements
  \begin{itemize}
  \item Dual-rail register cells with integrated completion
  \item Pipeline stage boundaries with Ko signal generation
  \item Reset and initialization circuitry
  \end{itemize}
\item Handshaking logic
  \begin{itemize}
  \item Ko signal generation and buffering
  \item Sleep signal distribution network
  \item Interface elements for synchronous boundaries
  \end{itemize}
\end{itemize}

Each cell undergoes rigorous characterization across process corners, including:
\begin{itemize}
\item Timing arcs for both rising and falling transitions
\item Power analysis in active and sleep modes
\item Area and physical design rules compliance
\item Statistical variation analysis
\end{itemize}

\subsection{Single-Rail Synthesis}
The initial synthesis stage leverages conventional synchronous synthesis techniques while accounting for MTNCL-specific requirements:

\begin{itemize}
\item Technology mapping optimizations
  \begin{itemize}
  \item Custom mapping rules for threshold gate selection
  \item Logic depth minimization for improved cycle time
  \item Gate sizing for optimal power/performance trade-off
  \end{itemize}
\item Logic optimization strategies
  \begin{itemize}
  \item Boolean optimization preserving dual-rail compatibility
  \item Path balancing for reduced completion time variation
  \item Common sub-expression elimination considering dual-rail costs
  \end{itemize}
\item Initial timing budgeting
  \begin{itemize}
  \item Delay allocation accounting for dual-rail expansion
  \item Critical path identification and optimization
  \item Setup margins for registration and completion detection
  \end{itemize}
\end{itemize}

\subsection{Dual-Rail Expansion}
The single-rail netlist undergoes automated transformation to dual-rail MTNCL format:

\begin{itemize}
\item Signal encoding conversion
  \begin{itemize}
  \item Automated DATA0/DATA1 rail generation
  \item Optimization of shared logic between rails
  \item Completion signal network synthesis
  \end{itemize}
\item Completion detection insertion
  \begin{itemize}
  \item Hierarchical completion tree generation
  \item Local completion group optimization
  \item Completion network balancing
  \end{itemize}
\item Sleep control implementation
  \begin{itemize}
  \item Sleep network topology optimization
  \item Buffer tree synthesis and sizing
  \item Power domain partitioning
  \end{itemize}
\end{itemize}

\subsection{Registration and Handshaking}
The final synthesis stage implements the asynchronous control infrastructure:

\begin{itemize}
\item Pipeline stage organization
  \begin{itemize}
  \item Register placement optimization
  \item Local completion group formation
  \item Critical path stage balancing
  \end{itemize}
\item Handshaking protocol implementation
  \begin{itemize}
  \item Ko signal generation and distribution
  \item Request/acknowledge synchronization
  \item Pipeline flow control optimization
  \end{itemize}
\item Physical design considerations
  \begin{itemize}
  \item Floorplanning for handshaking networks
  \item Clock tree replacement with sleep distribution
  \item Power grid design for sleep modes
  \end{itemize}
\end{itemize}

\subsection{Key Improvements Over Previous Approaches}
The proposed flow offers several advantages compared to traditional manual design:

\begin{itemize}
\item Automated optimization capabilities
  \begin{itemize}
  \item Systematic exploration of design space
  \item Concurrent optimization of multiple metrics
  \item Consistent application of design rules
  \end{itemize}
\item Enhanced reliability
  \begin{itemize}
  \item Comprehensive timing constraint generation
  \item Automated race condition prevention
  \item Systematic verification methodology
  \end{itemize}
\item Improved scalability
  \begin{itemize}
  \item Hierarchical design support
  \item Reusable methodology components
  \item Integration with existing EDA flows
  \end{itemize}
\end{itemize}

\section{Timing Constraints}
Essential timing constraints are required to ensure reliable operation of the synthesized MTNCL circuits. This section presents the key constraints necessary for proper circuit operation.

\subsection{Fundamental Timing Requirements}
The synthesis flow enforces three categories of timing constraints:

\begin{itemize}
\item Completion Detection Timing
  \begin{itemize}
  \item Maximum delay from sleep de-assertion to completion detection
  \item Minimum combinational logic delay paths
  \item Completion acknowledgment timing bounds
  \end{itemize}
\item Registration Timing
  \begin{itemize}
  \item Setup/hold requirements at registration elements
  \item Pipeline stage-to-stage timing relationships
  \item Reset timing for NULL propagation
  \end{itemize}
\item Handshaking Protocol Timing
  \begin{itemize}
  \item Ko signal propagation constraints
  \item Sleep signal distribution requirements
  \item Interface timing with adjacent stages
  \end{itemize}
\end{itemize}

These constraints are automatically generated and applied by the synthesis flow to prevent race conditions while enabling performance optimization. The constraints work in concert with the physical design tools to ensure timing closure through placement and routing.

\section{Results}
The effectiveness of the proposed synthesis flow and timing constraints was validated through the implementation of a 32-bit Montgomery Multiplier. This case study was chosen for its complexity and representative mix of control and datapath logic, making it an excellent benchmark for evaluating the methodology.

\subsection{Implementation Variants}
Three versions of the multiplier were implemented:
\begin{itemize}
\item Structural: Hand-crafted RTL using traditional MTNCL design techniques
\item Synthesized: Automated flow without optimization constraints
\item Optimized: Full flow with proposed timing constraints
\end{itemize}
All implementations were validated for functional correctness and reliability across process corners.

\subsection{Performance Analysis}
The optimized implementation achieved significant improvements:
\begin{itemize}
\item 35\% reduction in DATA-to-DATA cycle time
\item 28\% improvement in throughput
\item 40\% reduction in energy per operation
\end{itemize}
These gains were achieved while maintaining robustness against process variation and temperature effects.

\subsection{Area Impact}
The automated synthesis flow resulted in:
\begin{itemize}
\item 15\% increase in cell count vs. structural
\item 8\% increase in total area
\item Improved layout regularity and routability
\end{itemize}
The modest area overhead is justified by the significant performance and energy improvements.

\subsection{Reliability Verification}
Extensive verification confirmed:
\begin{itemize}
\item No timing violations across corners
\item Correct handling of all race conditions
\item Robust operation across voltage and temperature ranges
\end{itemize}
Static timing analysis and formal verification tools validated the effectiveness of the timing constraints.

\section{Conclusions}
This paper has presented a comprehensive methodology for synthesizing reliable MTNCL asynchronous circuits. The key contributions include:
\begin{itemize}
\item Identification and analysis of critical race conditions in MTNCL circuits
\item Development of a complete RTL-to-layout synthesis flow
\item Novel timing constraints that ensure reliability while enabling optimization
\item Validation through a complex Montgomery Multiplier implementation
\end{itemize}

The results demonstrate that automated synthesis of MTNCL circuits is not only feasible but can achieve better performance than traditional manual design approaches. The proposed timing constraints successfully prevent race conditions while enabling aggressive optimization, resulting in significant improvements in both performance and energy efficiency.

Critical insights gained during this work include:
\begin{itemize}
\item The importance of proper completion detection timing
\item The need for balanced handshaking constraints
\item Trade-offs between performance optimization and reliability
\end{itemize}

Future work directions include:
\begin{itemize}
\item Extension to more complex pipeline topologies
\item Integration with formal verification tools
\item Optimization for emerging process technologies
\end{itemize}

\end{document}
